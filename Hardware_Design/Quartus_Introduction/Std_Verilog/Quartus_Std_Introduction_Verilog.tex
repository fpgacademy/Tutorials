\documentclass[11pt, twoside, pdftex]{article}


\newcommand{\typeName}{Verilog}
\newcommand{\edition}{Standard}
%%%%%%%%%%%%%%%%%%%%%%%%%
% Preamble
% This include all the settings that we should use for the document

\newcommand{\PDFTitle}{Quartus\textsuperscript{\textregistered} Prime Introduction Using \typeName{} Designs}
\newcommand{\commonPath}{../../../Common}
\newcommand{\datePublished}{Mar 2022}

\newcommand{\versnum}{21.1} %version number quartus/AMP
\newcommand{\quartusname}{Quartus\textsuperscript{\textregistered} Prime}	
\newcommand{\textBar}{For \quartusname{} \versnum{}}
\newcommand{\thisyear}{2022 } %for copyright
\newcommand{\company}{FPGAcademy.org}
\newcommand{\longteamname}{FPGAcademy.org}
\newcommand{\teamname}{FPGAcademy}
\newcommand{\website}{FPGAcademy.org}

\newcommand{\productAcronym}{AMP}
\newcommand{\productNameShort}{Monitor Program}

\newcommand{\productNameMedTM}{Monitor Program}
\newcommand{\productNameMed}{Monitor Program}

%\newcommand{\headerLogoFilePath}[1]{#1/FPGAcademy.png}



\setlength\topmargin{-0.25in}
\setlength\headheight{0in}
\setlength\headsep{0.35in}
\setlength\textheight{8.5in}
\setlength\textwidth{7in}
\setlength\oddsidemargin{-0.25in}
\setlength\evensidemargin{-0.25in}
\setlength\parindent{0.25in}
\setlength\parskip{0in} 

\pdfpagewidth 8.5in
\pdfpageheight 11in

\input{\commonPath/Docs/listingsstyles}

%\usepackage[centering]{geometry}.
%%%%%%%%%%%%%%%%%%%%%%%%%%%%%%%%%%%%%%%%%%%%%%%%%%%
% Document Settings
\usepackage[labelsep=period]{caption}
% we can choose a better font later
%\usepackage{palatino}
\usepackage{fourier}
%\fontencoding{T1}
% include common used symbols
\usepackage{textcomp}
% add support for graphics
\usepackage{graphicx}
\usepackage[usenames, dvipsnames]{color}
% enable to draw thick or thin table hlines
\setlength{\doublerulesep}{\arrayrulewidth}
\usepackage{longtable}
\setlongtables
%\usepackage{array}
% It may be better to use PDFLaTeX as it can generate bookmarks for the
% document

% Add some useful packages
\usepackage{ae,aecompl}
\usepackage{epsfig,float,times}

% reset the font for section
\usepackage{sectsty}
%\allsectionsfont{\fontfamily{ptm}\selectfont}
\allsectionsfont{\usefont{OT1}{phv}{bc}{n}\selectfont}

% use compact space for sections
\usepackage[compact]{titlesec}
\titlespacing{\section}{0pt}{0.2in}{*0}
\titlespacing{\subsection}{0pt}{0.1in}{*0}
\titlespacing{\subsubsection}{0pt}{0.05in}{*0}

% fancyhdr header and footer customization
\usepackage{layout}
\usepackage{fancyhdr}
\pagestyle{fancy}
\fancyhead{}
\fancyhead[R]{\textit{\tiny{\textBar}}}
\fancyfoot{}
\fancyfoot[LO,
RE]{\textrm{\href{https://www.fpgacademy.org}{\small \longteamname}} \\ {\small \datePublished }}
\fancyfoot[RO, LE]{\small \thepage}
% two-side settings
%\fancyhead{} % clear all header fields
%\fancyfoot{} % clear all footer fields
%\fancyfoot[LE,RO]{\thepage}
\renewcommand{\headrulewidth}{2pt}
\renewcommand{\headrule}{{\color{blue} \hrule width\headwidth height\headrulewidth \vskip-\headrulewidth}}
\renewcommand{\footrulewidth}{0pt}

% Format the footer on page 1
\fancypagestyle{plain}{
\fancyhead{}
\fancyfoot{}
\fancyfoot[LO,
RE]{\textrm{\href{https://www.fpgacademy.org}{\small \longteamname}} \\ {\small \datePublished }}
\fancyfoot[RO, LE]{\small \thepage}
\renewcommand{\headrulewidth}{0pt}
}
% adjust some setting to try to make the figure stay in the same page with text
% Reference: 	http://www.cs.uu.nl/~piet/floats/node1.html
%   			http://mintaka.sdsu.edu/GF/bibliog/latex/floats.html
%   General parameters, for ALL pages:
\renewcommand{\topfraction}{0.9}	% max fraction of floats at top
\renewcommand{\bottomfraction}{0.8}	% max fraction of floats at bottom
%   Parameters for TEXT pages (not float pages):
\setcounter{topnumber}{3}
\setcounter{bottomnumber}{3}
\setcounter{totalnumber}{5}     % 2 may work better
\setcounter{dbltopnumber}{2}    % for 2-column pages
\renewcommand{\dbltopfraction}{0.9}	% fit big float above 2-col. text
\renewcommand{\textfraction}{0.07}	% allow minimal text w. figs
%   Parameters for FLOAT pages (not text pages):
\renewcommand{\floatpagefraction}{0.7}	% require fuller float pages
% N.B.: floatpagefraction MUST be less than topfraction !!
\renewcommand{\dblfloatpagefraction}{0.7}	% require fuller float pages
%%%%%%%%%%%%%%%%%%%%%%%%%%%%%%%%%%%%%%%%%%%%%%%%%%%
% remember to use [htp] or [htpb] for placement
%%%%%%%%%%%%%%%%%%%%%%%%%%%%%%%%%%%%%%%%%%%%%%%%%%%

% set no indent for paragraph
\setlength{\parindent}{0em}
\addtolength{\parskip}{11pt}
\newcommand{\compact}{[topsep=0pt]}
% use this package to reduce space
\usepackage{enumitem}
\usepackage{multirow}
\usepackage{rotating}
\usepackage{pifont}
\usepackage{dingbat}
\newcommand{\itemsecond}{$\circ$}
%
%%%%%%%%%%%%%%%%%%
\date{}
\author{}
%%%%%%%%%%%%%%%%%%
\newcommand{\de}{DE-series}
\newcommand{\up}{FPGAcademy}
\newcommand{\fabric}{Avalon Switch Fabric}
\newcommand{\TODO}[1]{\textcolor{red}{\textbf{TODO}: #1}}
\def\registered{{\ooalign{\hfil\raise .00ex\hbox{\scriptsize R}\hfil\crcr\mathhexbox20D}}}

% enable url and reference(bookmarks) in pdf
\usepackage{url}
\usepackage[pdftex, colorlinks]{hyperref}
\hypersetup{%
pdftitle={\PDFTitle},
linkcolor=blue,
hyperindex=true,
pdfauthor={\longteamname},
pdfkeywords={FPGAcademy, Academic Program, Example System},
bookmarksnumbered,
bookmarksopen=false,
filecolor=blue,
pdfstartview={FitH},
urlcolor=blue,
plainpages=false,
pdfpagelabels=true,
linkbordercolor={1 1 1} %no color for link border
}%
%%%%%%%%%%%%%%%%%%%%%%%%%%%%%%%%%%%%%%%%%%%%%%%%%%%
\setlength{\fboxsep}{0.7pt}
\setlength{\fboxrule}{0.5pt}




%%%%%%%%%%%%%%%%%%%%%%%%%
% Add title
\newcommand{\doctitle}{Quartus\textsuperscript{\textregistered} Prime Introduction \\ Using \typeName{} Designs}
\newcommand{\dochead}{Quartus\textsuperscript{\textregistered} Prime Introduction Using \typeName{} Designs}
% Usually no need to change these two lines
\title{\fontfamily{phv}\selectfont{\doctitle} }
\chead{ \small{\textsc{\bfseries \dochead} } }
% Customizations
%%%%%%%%%%%%%%%%%%%%%%%%%
% Allows multiple figures per page

\renewcommand\floatpagefraction{.9}
\renewcommand\topfraction{.9}
\renewcommand\bottomfraction{.9}
\renewcommand\textfraction{.1}   
\setcounter{totalnumber}{50}
\setcounter{topnumber}{50}
\setcounter{bottomnumber}{50}
\raggedbottom

%%%%%%%%%%%%%%%%%%%%%%%%%

%%%%%%%%%%%%%%%%%%%%%%%%%
% Document Begings
\begin{document}

%%%%%%%%%%%%%%%%%%%%%%%%%
% Header Logo and Title
\begin{table}

    \centering
    \begin{tabular}{p{5cm}p{4cm}}
        \hspace{-3cm}
        &
        \raisebox{1\height}{\parbox[h]{0.5\textwidth}{\Large\fontfamily{phv}\selectfont{\textsf{\doctitle}}}}
    \end{tabular}
    \label{tab:logo}
\end{table}

\colorbox[rgb]{0,0.384,0.816}{\parbox[h]{\textwidth}{\color{white}\textsf{\textit{\textBar}}}}

\thispagestyle{plain}
%%%%%%%%%%%%%%%%%%%%%%%%%

%%%%%%%%%%%%%%%%%%%%%%%%%
% Table of Contents
{\tiny \tableofcontents}
\newpage
%%%%%%%%%%%%%%%%%%%%%%%%%

%%%%%%%%%%%%%%%%%%%%%%%%%
% Introduction
\section{Introduction}


This tutorial presents an introduction to the Quartus\textsuperscript{\textregistered} Prime CAD system.
It gives a general overview of a typical CAD flow for designing circuits that are
implemented by using FPGA devices, and shows how this flow is realized in
the Quartus Prime software. The design process is illustrated by giving step-by-step
instructions for using the Quartus Prime software to implement a very simple circuit
in an Intel\textsuperscript{\textregistered} FPGA device.

The Quartus Prime system includes full support for all of the popular methods of
entering a description of the desired circuit into a CAD system. This tutorial
makes use of the \typeName{} design entry method, in which the user specifies the desired
circuit in the \typeName{} hardware description language. 
Three versions of this tutorial are available; one
uses the Verilog hardware description language, another uses the VHDL hardware description language, and the third is based on defining
the desired circuit in the form of a schematic diagram.

The last step in the design process involves configuring the designed circuit
in an actual FPGA device. To show how this is done, it is assumed that the user has access
to the Intel DE-series Development and Education board connected to 
a computer that has Quartus Prime software installed. 
A reader who does not have access to the DE-series board will still find the tutorial useful
to learn how the FPGA programming and configuration task is performed.

The screen captures in the tutorial were obtained using the 
Quartus Prime version \versnum \ \edition{} Edition; other versions of the software may be slightly different.

%%%%%%%%%%%%%%%%%%%%%%%%%

%%%%%%%%%%%%%%%%%%%%%%%%%
% Background
\newpage


\section{Background}
Computer Aided Design (CAD) software makes it easy to implement a desired logic
circuit by using a programmable logic device, such as a Field-Programmable 
Gate Array (FPGA) chip. A typical FPGA CAD flow is illustrated in Figure~\ref{fig:1}. 

\begin{figure}[H]
   \begin{center}
      \includegraphics[scale=1]{figures/figure1.png}
   \caption{Typical CAD flow.} 
	 \label{fig:1}
	 \end{center}
\end{figure}

The CAD flow involves the following steps:
\begin{itemize}
\item {\bf Design Entry} -- the desired circuit is specified either by means of
a schematic diagram, or by using a hardware description language, 
such as Verilog or VHDL
\item {\bf Synthesis} -- the entered design is synthesized into a circuit
that consists of the logic elements (LEs) provided in the FPGA chip
\item {\bf Functional Simulation} -- the synthesized circuit is tested to
verify its functional correctness; this simulation does not take into account
any timing issues
\item {\bf Fitting} -- the CAD Fitter tool determines the placement of the LEs 
defined in the netlist into the LEs in an actual FPGA chip; it also 
chooses routing wires in the chip to make the required connections 
between specific LEs 
\item {\bf Timing Analysis} -- propagation delays along the various paths
in the fitted circuit are analyzed to provide an indication of the expected
performance of the circuit
\item {\bf Timing Simulation} -- the fitted circuit is tested to verify both
its functional correctness and timing
\item {\bf Programming and Configuration} -- the designed circuit is implemented
in a physical FPGA chip by programming the configuration switches that configure
the LEs and establish the required wiring connections
\end{itemize}

\noindent
This tutorial introduces the basic features of the Quartus Prime software. 
It shows how the software can be used to design and implement a circuit specified by
using the \typeName{} hardware description language.
It makes use of the graphical user interface to invoke the Quartus Prime commands.
Doing this tutorial, the reader will learn about:
\begin{itemize}
\item Creating a project
\item Design entry using \typeName{} code
\item Synthesizing a circuit specified in \typeName{} code
\item Fitting a synthesized circuit into an FPGA
\item Assigning the circuit inputs and outputs to specific pins on the FPGA
\item Simulating the designed circuit
\item Programming and configuring an FPGA chip
\end{itemize}

%%%%%%%%%%%%%%%%%%%%%%%%%

%%%%%%%%%%%%%%%%%%%%%%%%%
% Getting Started
\section{Getting Started}

\noindent 
Each logic circuit, or subcircuit, being designed with Quartus Prime software is
called a {\it project}. The software works on one project at a time
and keeps all information for that project in a single directory (folder) in
the file system.
To begin a new logic circuit design, the first step is
to create a directory to hold its files. 
To hold the design files for this tutorial, we will use a directory 
{\it introtutorial}.
The running example for this tutorial is a simple circuit for two-way light control.

Start the Quartus Prime software. You should see a display
similar to the one in Figure~\ref{fig:2}. This display consists of several windows that 
provide access to all the features of 
Quartus Prime software, which the user selects with the computer mouse.
Most of the commands provided by Quartus Prime software can be accessed by using a set of
menus that are located below the title bar. For
example, in Figure~\ref{fig:2} clicking the left mouse button on the menu
named {\sf File} opens the menu shown in Figure~\ref{fig:3}. Clicking the
left mouse button on the entry {\sf Exit} exits
from Quartus Prime software. In general, whenever the mouse is used to select
something, the {\it left} button is used. Hence we will not normally
specify which button to press. In the few cases when it is
necessary to use the {\it right} mouse button, it will be specified explicitly. 

\begin{figure}[H]
   \begin{center}
      \includegraphics[scale=0.4]{figures/figure2.png}
   \caption{The main Quartus Prime display.} 
	 \label{fig:2}
	 \end{center}
\end{figure}

\begin{figure}[H]
   \begin{center}
      \includegraphics[scale=0.45]{figures/figure3.png}
   \caption{An example of the File menu.} 
	 \label{fig:3}
	 \end{center}
\end{figure}

\newpage
For some commands it is necessary to access two or more menus in sequence.
We use the convention {\sf Menu1 $>$ Menu2 $>$ Item} to indicate that 
to select the desired command 
the user should first click the left mouse button on {\sf Menu1}, then 
within this menu click on {\sf Menu2}, and then
within {\sf Menu2} click on {\sf Item}. For example, 
{\sf File $>$ Exit} uses the mouse to exit from the system.
Many commands can be invoked by clicking on an icon displayed in 
one of the toolbars. To see the command associated with an icon, position the mouse
over the icon and the command name will be shown in the status bar at the bottom of the screen.

%%%%%%%%%%%%%%%%%%%%%%%%%

%%%%%%%%%%%%%%%%%%%%%%%%%
% Quartus Online Help
\subsection{Quartus\textsuperscript{\textregistered} Prime Online Help}


The Quartus Prime software provides comprehensive online documentation that addresses
many of the questions that may arise when using the software. The documentation is accessed
from the {\sf Help} menu.  To get some idea of the extent of documentation provided,
it is worthwhile for the reader to browse through the {\sf Help} menu.

The user can quickly search through the {\sf Help} topics by using the 
search box in the top right corner of the main Quartus display.
Another method, context-sensitive help, is provided for quickly finding documentation for
specific topics. While using most applications, pressing the {\sf F1} function
key on the keyboard opens a Help display that
shows the commands available for the application. 

%%%%%%%%%%%%%%%%%%%%%%%%%

%%%%%%%%%%%%%%%%%%%%%%%%%
% Starting A New Project
\input{../shared_sections/section_starting_new_project_std.tex}
%%%%%%%%%%%%%%%%%%%%%%%%%

%%%%%%%%%%%%%%%%%%%%%%%%%
% Design Entry Using VHDL
\section{Design Entry Using Verilog Code}


\noindent
As a design example, we will use the two-way light controller circuit shown in 
Figure~\ref{fig:11}. The circuit can be used to control a single light from either of the
two switches, $x_1$ and $x_2$, where a closed switch corresponds to the logic value 1.
The truth table for the circuit is also given in the figure. Note that
this is just the Exclusive-OR function of the inputs $x_1$ and $x_2$,
but we will specify it using the gates shown.

\begin{figure}[H]
   \begin{center}
      \includegraphics[scale=1]{figures/figure11.png}
   \caption{The light controller circuit.} 
	 \label{fig:11}
	 \end{center}
\end{figure}

The required circuit is described by the Verilog code in Figure~\ref{fig:12}.
Note that the Verilog module is called {\it light} to match the name given in 
Figure~\ref{fig:4}, which was specified when the project was created.
This code can be typed into a file by using any text editor
that stores ASCII files, or by using the Quartus Prime text editing facilities.
While the file can be given any name, it is a common designers' practice to
use the same name as the name of the top-level Verilog module.
The file name must include the extension $v$, which indicates a Verilog
file. So, we will use the name {\it light.v}.

\lstset{language=Verilog}
\begin{figure}[H]
\begin{center}
\begin{lstlisting}
module light (x1, x2, f);
	input x1, x2;
	output f;
	assign f = (x1 & ~x2)|(~x1 & x2);
endmodule 
\end{lstlisting}
\end{center}
\vspace{-0.33in}
	\caption{Verilog code for the circuit in Figure 11.}
	\label{fig:12}
\end{figure}

\subsection{Using the Quartus\textsuperscript{\textregistered}	 Prime Text Editor}

\noindent 
This section shows how to use the Quartus Prime Text Editor.
You can skip this section if you prefer to use some other text editor
to create the Verilog source code file, which we will name {\it light.v}. 

Select {\sf File $>$ New} to get the window in Figure~\ref{fig:13}, 
choose {\sf Verilog HDL File}, and click {\sf OK}. 
This opens the Text Editor window. 
The first step is to specify a name
for the file that will be created. Select {\sf File $>$ Save As}
to open the pop-up box depicted in Figure~\ref{fig:14}. 
In the box labeled {\sf Save as type} choose {\sf Verilog HDL File}.
In the box labeled {\sf File name} type {\it light}.
Put a checkmark in the box {\sf Add file to current project}.
Click {\sf Save}, which puts the file into the directory
{\it introtutorial} and leads to the Text Editor window shown
in Figure~\ref{fig:15}. 
Enter the Verilog code in Figure~\ref{fig:12}
into the Text Editor and save the file by typing {\sf File $>$ Save}, or by typing 
the shortcut {\sf Ctrl-s}.

Most of the commands available in the Text Editor are self-explanatory. 
Text is entered at the {\it insertion point}, which is indicated by a thin
vertical line. The insertion point can be moved either by using the
keyboard arrow keys or by using the mouse. Two features of 
the Text Editor are especially convenient for typing Verilog
code. First, the editor can display different types of Verilog
statements in different colors, which is the default choice. 
Second, the editor can automatically
indent the text on a new line so that it matches the previous line. 
Such options can be controlled by the settings 
in {\sf Tools $>$ Options $>$ Text Editor}.

\begin{figure}[H]
   \begin{center}
      \includegraphics[scale=0.65]{figures/figure13.png}
   \caption{Choose to prepare a Verilog file.} 
	 \label{fig:13}
	 \end{center}
\end{figure}

\begin{figure}[H]
   \begin{center}
      \includegraphics[scale=0.55]{figures/figure14.png}
   \caption{Name the file.} 
	 \label{fig:14}
	 \end{center}
\end{figure}

\begin{figure}[H]
   \begin{center}
      \includegraphics[scale=0.40]{figures/figure15.png}
   \caption{Text Editor window.} 
	 \label{fig:15}
	 \end{center}
\end{figure}

\subsubsection{Using Verilog Templates}

The syntax of Verilog code is sometimes difficult for a
designer to remember. To help with this issue, the Text Editor
provides a collection of {\it Verilog templates}. The templates provide
examples of various types of Verilog statements, such as a {\bf module}
declaration, an {\bf always} block, and assignment statements. 
It is worthwhile to browse through the templates
by selecting {\sf Edit $>$ Insert Template $>$ Verilog HDL} to become 
familiar with this resource.

\subsection{Adding Design Files to a Project}

As we indicated when discussing Figure~\ref{fig:6}, you can tell Quartus Prime software
which design files it should use as part of the current project.
To see the list of files already included in the {\it light} project,
select {\sf Assignments $>$ Settings}, which leads to the window in Figure~\ref{fig:16}.
As indicated on the left side of the figure, click on the item {\sf Files}.
An alternative way of making this selection is to choose
{\sf Project $>$ Add/Remove Files in Project}.

If you used the Quartus Prime Text Editor to create the file and checked
the box labeled {\sf Add file to current project},
as described in Section 5.1, then the {\it light.v}
file is already a part of the project and will be listed in
the window in Figure~\ref{fig:16}.
Otherwise, the file must be added to the project. 
So, if you did not use the Quartus Prime Text Editor, then place a copy of the 
file {\it light.v}, which you created using some other text editor, into 
the directory {\it introtutorial}.
To add this file to the project, click on the {\sf ...} button next to the 
box labeled {\sf File name} in
Figure~\ref{fig:16} to get the pop-up window in Figure~\ref{fig:17}.
Select the {\it light.v} file and click {\sf Open}.
The selected file is now indicated in the {\sf File name} box in Figure~\ref{fig:16}. 
Click {\sf Add} then {\sf OK} to include the {\it light.v} file in the project.
We should mention that in many cases the Quartus Prime software is able to 
automatically find the right files to use for each entity 
referenced in Verilog code, even if the file has not been 
explicitly added to the project. However, for complex projects that
involve many files it is a good design practice to specifically
add the needed files to the project, as described above.

\begin{figure}[H]
   \begin{center}
      \includegraphics[scale=0.55]{figures/figure16.png}
   \caption{Settings window.} 
	 \label{fig:16}
	 \end{center}
\end{figure}

\begin{figure}[H]
   \begin{center}
      \includegraphics[scale=0.65]{figures/figure17.png}
   \caption{Select the file.} 
	 \label{fig:17}
	 \end{center}
\end{figure}

%%%%%%%%%%%%%%%%%%%%%%%%%

%%%%%%%%%%%%%%%%%%%%%%%%%
% Compiling Designed Circuit
\input{../shared_sections/section_compiling_designed_circuit_verilog.tex}
%%%%%%%%%%%%%%%%%%%%%%%%%

%%%%%%%%%%%%%%%%%%%%%%%%%
% Pin Assignments
\section{Pin Assignment}

During the compilation process described above, the Compiler was free to choose any
pins on the selected FPGA device to serve as inputs and outputs. However, an FPGA board
has hardwired connections between the pins of the FPGA chip and the other components that
are on the board.
We will use two toggle switches, labeled $SW_0$ and $SW_1$, to provide the
external inputs, $x_1$ and $x_2$, to our example circuit. These switches are connected
to the FPGA pins listed in Table \ref{tab:pinassign}. We will connect the output $f$ to a
light-emitting diode on your DE-series board. For the DE2-115 we will use a green LED: $LEDG_0$.
On the DE0-CV, DE1-SoC, DE10-Lite and DE10-Standard we will use $LEDR_0$. On the DE0-Nano and DE0-Nano-SoC, we will use $LED_0$
The FPGA pin assignment for the LEDs are listed in Table~\ref{tab:pinassign}.

\begin{table}[H]
\centering
\begin{tabular}{| c | c | c | c |}
\hline
Component & $SW_0$ & $SW_1$ & {\it LEDG}$_0$, {\it LED}$_0$, or {\it LEDR}$_0$ \\
\hline
DE0-CV & PIN$\_$U13 & PIN$\_$V13 & PIN$\_$AA2 \\
\hline
DE0-Nano & PIN$\_$M1 & PIN$\_$T8 & PIN$\_$A1 \\
\hline
DE0-Nano-SoC & PIN$\_$L10 & PIN$\_$L9 & PIN$\_$W15 \\
\hline
DE2-115 & PIN$\_$AB28 & PIN$\_$AC28 & PIN$\_$E21 \\
\hline
DE1-SoC & PIN$\_$AB12 & PIN$\_$AC12 & PIN$\_$V16 \\
\hline
DE10-Lite & PIN$\_$C10 & PIN$\_$C11 & PIN$\_$A8 \\
\hline
DE10-Standard & PIN$\_$AB30 & PIN$\_$AB28 & PIN$\_$AA24 \\
\hline
DE10-Nano & PIN$\_$Y24 & PIN$\_$W24 & PIN$\_$W15 \\
\hline
\end{tabular}
 
\caption{DE-Series Pin Assignments}
\label{tab:pinassign}
\end{table}

\begin{figure}[H]
   \begin{center}
      \includegraphics[scale=0.65]{figures/figure22.png}
   \caption{The Assignment Editor window.} 
	 \label{fig:22}
	 \end{center}
\end{figure}

Pin assignments can be made by using the {\it Assignment Editor}. 
Select {\sf Assignments $>$ Assignment Editor} to reach the window in Figure~\ref{fig:22}
(shown here as a detached window).
In the {\sf Category} drop-down menu select {\sf All}. Click on the {\sf $<$$<$new$>$$>$} button
located near the top left corner to make a new item appear in the table. Double-click the box
under the column labeled {\sf To} so that the {\sf Node Finder} button 
\includegraphics[scale=0.65]{figures/icon7.png}
appears. Click on the button (not the drop down arrow) to reach the window in 
Figure~\ref{fig:23}. Click on \includegraphics[scale=0.6]{figures/icon12.png} and 
\includegraphics[scale=0.6]{figures/icon15.png} to show or hide more search options.
In the {\sf Filter} drop-down menu select {\sf Pins: all}. Then, click the {\sf List} 
button to display the input and output pins to be assigned: $f$, $x1$, and $x2$.
Click on $x1$ as the first pin to be assigned and click the {\sf $>$} button; this action will 
enter $x1$ in the {\sf Nodes Found} box.  
Click {\sf OK}. The $x1$ node will now appear in the box 
under the column labeled {\sf To} in the Assignment Editor window. Alternatively, the node 
name can be entered directly by 
double-clicking the box under the {\sf To} column and typing in the node name.

Now double-click on the box to the right of this new $x1$ entry, in the column
labeled {\sf Assignment Name}, to open the drop-down menu in Figure~\ref{fig:24}. Scroll 
down and select {\sf Location (Accepts wildcards/groups)}. Instead of scrolling down the menu 
to find the desired item, you can alternatively type the first letter of the item in the 
{\sf Assignment Name} box. In this case the desired item happens to be the first item 
beginning with {\sf L}. Finally, double-click the box in the column labeled {\sf Value}.
Type the pin assignment corresponding to $SW_0$ for your DE-series board, as listed in 
Table \ref{tab:pinassign}.

Use the same procedure to assign input $x2$ and output $f$ to the appropriate pins listed in
Table \ref{tab:pinassign}. An example using a DE1-SoC board is shown in Figure~\ref{fig:25}.
To save the assignments made, choose {\sf File $>$ Save}. You can also simply close 
the Assignment Editor window, in which case a pop-up box will ask if you want to save
the changes to assignments; click {\sf Yes}. Finally, execute the {\sf Processing $>$ 
Start Compilation} command to create a new circuit that makes use of your pin assignments. 

\begin{figure}[H]
   \begin{center}
      \includegraphics[scale=0.65]{figures/figure23.png}
   \caption{The Node Finder displays the input and output names.} 
	 \label{fig:23}
	 \end{center}
\end{figure}

\begin{figure}[H]
   \begin{center}
      \includegraphics[scale=0.75]{figures/figure24.png}
   \caption{The available assignment names for a DE-series board.} 
	 \label{fig:24}
	 \end{center}
\end{figure}

\begin{figure}[H]
   \begin{center}
      \includegraphics[scale=0.65]{figures/figure25.png}
   \caption{The complete assignment.} 
	 \label{fig:25}
	 \end{center}
\end{figure}

When you make assignments related to your project, such as the pin assignments described
above, or the device assignments made earlier in the tutorial, Quartus Prime stores these 
assignments in a special type of file associated with your project. This file is called 
a {\it quartus settings file} ({\it qsf}), and in our case would be named {\it light.qsf}. 
In this file the pin assignments created above (using a DE1-SoC board as an example) would
be included as

\begin{center}
\begin{verbatim}
	set_location_assignment PIN_AB12 -to x1
	set_location_assignment PIN_AC12 -to x2
	set_location_assignment PIN_V16 -to f
\end{verbatim}
\end{center}

A useful Quartus Prime feature allows the user to import into a project the assignments
contained in a quartus settings file, rather than creating them manually using the 
Assignment Editor.  Importing pin assignments from an existing {\it qsf} file can be a
convenient way of making these assignments, rather than following the (somewhat tedious)
process described above.   

\noindent
You can import pin assignments by choosing {\sf Assignments $>$ Import Assignments}. 
This command opens the dialogue in Figure~\ref{fig:27}, allowing you to select a file to import. 
 
\begin{figure}[H]
   \begin{center}
      \includegraphics[scale=0.65]{figures/figure27.png}
   \caption{Importing the pin assignment.} 
	 \label{fig:27}
	 \end{center}
\end{figure}


% Additional to pin assignments sections.
For convenience when using large designs, all relevant pin assignments for the 
DE-series board are given in individual files. For example, the DE1-SoC pin assignments 
can be found in the {\it DE1\_SoC.qsf} file, which is available from Intel's FPGA University
Program website.
This file uses the names found in the {\it DE1-SoC User Manual}.
If we wanted to make the pin assignments for our example circuit by importing
this file, then we would have to use the same names in our 
Block Diagram/Schematic design file;
namely, {\it SW[0]}, {\it SW[1]} and {\it LEDG[0]} for 
{\it x1}, {\it x2} and {\it f}, respectively.
Since these signals are specified in the {\it DE1\_SoC.qsf} file
as elements of vectors {\it SW} and {\it LEDG}, we must refer to them in the same
way in our design file. For example, in the {\it DE1\_SoC.qsf} 
file the 10 toggle switches are called {\it SW[9]} to {\it SW[0]}.
In a design file they can also be referred to as a vector {\it SW[9..0]}.
%%%%%%%%%%%%%%%%%%%%%%%%%

%%%%%%%%%%%%%%%%%%%%%%%%%
% Programming and Configuring
\newpage
\section{Programming and Configuring the FPGA Device}

The FPGA device must be programmed to implement the designed 
circuit. The required configuration file is generated by the Quartus Prime 
Compiler's Assembler module. Each DE-series board allows the configuration to 
be done in two different ways, known as JTAG* and AS modes.
The configuration data is transferred from the host computer (which runs the 
Quartus Prime software) to the board by means of a cable that connects 
a USB port on the host computer to the {\it USB-Blaster} connector on the board.
To use this connection, it is necessary to have the USB-Blaster (II) driver 
installed. If this driver is not already installed, consult the 
tutorial {\it Getting Started with the Terasic DE-Series Boards},
available on {\small \href{https://www.fpgacademy.org/tutorials.html} {FPGAcademy.org}},
for information about installing the driver. You can also search on the Internet for a
topic such as ``Installing USB Blaster driver''.
Before using the board, make sure that the USB cable is properly connected
and that the board is powered on.
 
In the JTAG mode, the configuration data is loaded directly
into the FPGA device. The acronym JTAG stands for Joint Test Action Group. 
This group defined a simple way for testing digital circuits and loading data 
into them, which became an IEEE* standard. If the FPGA is configured in 
this manner, it will retain its 
configuration as long as the power remains turned on. 
The configuration information is lost when the power is turned off.
The second possibility is to use the Active Serial (AS) mode.
In this case, a configuration device that includes some flash memory is used 
to store the configuration data. Quartus Prime software places the configuration 
data into the configuration device on the DE-series board. Then, this data is loaded 
into the FPGA upon power-up or reconfiguration.
Thus, the FPGA need not be configured by the Quartus Prime software if the power 
is turned off and on. 
The choice between the two modes is made by switches on the DE-series 
board. Consult your manual for the location of this switch on your DE-series board. The boards should be set to JTAG mode by default.
This tutorial discusses only the JTAG programming mode.

\subsection{JTAG* Programming for the DE1-SoC Board, DE0-Nano-SoC, DE10-Nano, and DE10-Standard}

For the DE1-SoC Board, DE0-Nano-SoC, DE10-Nano, and DE10-Standard boards, 
the following steps should be used for programming.  Select {\sf Tools $>$ Programmer} to reach 
the window in Figure~\ref{fig:SoC1}. If the {\it SOCVHPS} device shown in the figure is missing, 
then you need to add it. Click on the {\sf Add Device} menu, and then look for the {\it SOCVHPS}
device under the \textit{Soc Series V} family.
Now it is necessary to specify the programming hardware and 
the mode that should be used. If not already chosen by default, 
select JTAG in the {\sf Mode} box.  The required programming hardware is called {\it DE-SoC}. If
this hardware is not chosen by default as the programming
hardware, then press the {\sf Hardware Setup...} button and select the {\it DE-SoC} in the 
window that pops up, as shown in Figure~\ref{fig:SoC2}.\footnote{In some versions of the
Microsoft Windows operating system the drivers provided with older versions of the Quartus
Prime software might fail to install. In such cases, no programming Hardware will be available to
the Quartus Programmer. One approach to solving such an issue is to download and install the 
drivers that are provided with a more recent version of the Quartus Prime software.
If you install a version of Quartus Prime into a folder named
\texttt{Qdir}, then the drivers that come with this version can be found in the sub-folder
\texttt{Qdir/quartus/drivers}. The driver for the DE-SoC hardware is called {\it
USB-Blaster-II}.}  

Observe that the configuration file {\it light.sof} in directory {\it output\_files} is 
listed in the window in
Figure~\ref{fig:SoC1}. If the file is not already listed, then click {\sf Add File} 
and select it.  This is a binary file produced by the Compiler's Assembler module, 
which contains the data needed to configure the FPGA device.
The extension {\it .sof} stands for SRAM Object File.
Ensure the {\sf Program/Configure} box is checked.
This setting is used to select the FPGA in the Cyclone V SoC chip for programming.

Press {\sf Start} in the Programmer to program the FPGA device.  An LED on the board will 
light up while the FPGA device is being programmed. 
If you see an error reported by Quartus Prime software indicating that
programming {\it failed}, then check to ensure that the board is properly powered on.
Also, as mentioned above, if the {\it SOCVHPS} device is not shown as in Figure~\ref{fig:SoC1}, 
click {\sf Add Device $>$ SoC Series V $>$ SOCVHPS}, and then click {\sf OK} to add it.
Some boards use the device order shown in Figure~\ref{fig:SoC1}, and others use the
opposite order. If programming of your board failed, try reversing the order
from what is shown in the figure, by clicking on a device and then clicking 
\includegraphics[scale=.55]{figures/icon14.png} or 
\includegraphics[scale=.55]{figures/icon13.png}.

\begin{figure}[H]
   \begin{center}
      \includegraphics[width=0.65\textwidth]{figures/figureSoC1.png}
   \caption{The Programmer window.} 
	 \label{fig:SoC1}
	 \end{center}
\end{figure}

\begin{figure}[H]
   \begin{center}
      \includegraphics[scale=0.65]{figures/figureSoC2.png}
   \caption{The Hardware Setup window.} 
	 \label{fig:SoC2}
	 \end{center}
\end{figure}

\subsection{JTAG* Programming for the DE0-CV, DE0-Nano, DE10-Lite, and DE2-115 Boards}

For the DE0-CV, DE0-Nano, DE10-Lite, and DE2-115 Boards, the
programming and configuration task is performed as follows. 
Select {\sf Tools $>$ Programmer} to reach the window in Figure~\ref{fig:38}.
Here it is necessary to specify the programming hardware and 
the mode that should be used. If not already chosen by default, 
select JTAG in the Mode box.
Also, if the USB-Blaster is not chosen by default, press the 
{\sf Hardware Setup...} button and select the USB-Blaster in the window
that pops up, as shown in Figure~\ref{fig:39}. \footnote{In some versions of the
Microsoft Windows operating system the USB-Blaster driver provided with older versions of the 
Quartus Prime software might fail to install. In such cases, no programming Hardware will be 
available to the Quartus Programmer. One approach to solving such an issue is to download and 
install the USB-Blaster driver that is provided with a more recent version of the Quartus 
Prime software. If you install a version of Quartus Prime into a folder named
\texttt{Qdir}, then the drivers that come with this version can be found in the sub-folder
\texttt{Qdir/quartus/drivers}.}  

\begin{figure}[H]
   \begin{center}
      \includegraphics[scale=0.65]{figures/figure38.png}
   \caption{The Programmer window.} 
	 \label{fig:38}
	 \end{center}
\end{figure}

Observe that the configuration file {\it light.sof} is listed in the window in
Figure~\ref{fig:38}. If the file is not already listed, then click {\sf Add File}
and select it.
This is a binary file produced by the Compiler's Assembler module, 
which contains the data needed to configure the FPGA device.
The extension {\it .sof} stands for SRAM Object File.
Ensure the {\sf Program/Configure} check box is ticked, as shown in Figure~\ref{fig:38}.

\begin{figure}[H]
   \begin{center}
      \includegraphics[scale=0.65]{figures/figure39.png}
   \caption{The Hardware Setup window.} 
	 \label{fig:39}
	 \end{center}
\end{figure}

Now, press {\sf Start} in the window in Figure~\ref{fig:38}.
An LED on the board will light up corresponding to the programming operation.
If you see an error reported by Quartus Prime software indicating that
programming failed, then check to ensure that the board is properly powered on.


%%%%%%%%%%%%%%%%%%%%%%%%%

%%%%%%%%%%%%%%%%%%%%%%%%%
% Testing Designed Circuit
\section{Testing the Designed Circuit}

Before implementing the designed circuit in the FPGA chip on the DE-series board, it is
prudent to simulate it to ascertain its correctness. While not covered in this tutorial,
users may use software such as {\it ModelSim} or other simulation environments to test the 
circuit in simulation. Simulation of a circuit often provides a comprehensive view of the 
circuit's functionality, and can help users easily find bugs within the circuit's logic 
without having to touch hardware.

Having downloaded the configuration data into the FPGA device, you can now
test the implemented circuit.  Try all four valuations of the input variables
$x_1$ and $x_2$, by setting the corresponding states of the switches
$SW_1$ and $SW_0$. Verify that the circuit implements the truth table
in Figure~\ref{fig:11}.

If you want to make changes in the designed circuit, first close the Programmer window.
Then make the desired changes in the \typeName{} design file, compile the circuit,
and program the board as explained above.

%%%%%%%%%%%%%%%%%%%%%%%%%

%%%%%%%%%%%%%%%%%%%%%%%%%
% Copyright

%\newcommand{\datePublished}{Mar 2022}

\newcommand{\versnum}{21.1} %version number quartus/AMP
\newcommand{\quartusname}{Quartus\textsuperscript{\textregistered} Prime}	
\newcommand{\textBar}{For \quartusname{} \versnum{}}
\newcommand{\thisyear}{2022 } %for copyright
\newcommand{\company}{FPGAcademy.org}
\newcommand{\longteamname}{FPGAcademy.org}
\newcommand{\teamname}{FPGAcademy}
\newcommand{\website}{FPGAcademy.org}

\newcommand{\productAcronym}{AMP}
\newcommand{\productNameShort}{Monitor Program}

\newcommand{\productNameMedTM}{Monitor Program}
\newcommand{\productNameMed}{Monitor Program}

%\newcommand{\headerLogoFilePath}[1]{#1/FPGAcademy.png}



%%%%%%%%%%%%%%%%%%%%%%%%%%%%%%%%%%%%%%%%
%%% FPGAcademy Copyright Information %%%
%%%%%%%%%%%%%%%%%%%%%%%%%%%%%%%%%%%%%%%%

%Always put the copyright on a new page (clear page), with some vertical space from top
\clearpage
\vspace{1in}

\noindent

Copyright {\copyright} FPGAcademy.org. All rights reserved. FPGAcademy and the FPGAcademy logo are trademarks of  FPGAcademy.org.  This document is being provided on an ``as-is'' basis and as an accommodation and therefore all warranties, representations or guarantees of any kind (whether express, implied or statutory) including, without limitation, warranties of merchantability, non-infringement, or fitness for a particular purpose, are specifically disclaimed.

%FPGAcademy assumes no responsibility or liability arising out of the application or use of any information,  product,  or  service  described  herein  except  as  expressly  agreed  to  in  writing  by  FPGAcademy.



**Other names and brands may be claimed as the property of others.



%%%%%%%%%%%%%%%%%%%%%%%%%

\end{document}
% Document Ends
%%%%%%%%%%%%%%%%%%%%%%%%%
