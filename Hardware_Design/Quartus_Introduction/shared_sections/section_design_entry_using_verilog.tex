\section{Design Entry Using Verilog Code}


\noindent
As a design example, we will use the two-way light controller circuit shown in 
Figure~\ref{fig:11}. The circuit can be used to control a single light from either of the
two switches, $x_1$ and $x_2$, where a closed switch corresponds to the logic value 1.
The truth table for the circuit is also given in the figure. Note that
this is just the Exclusive-OR function of the inputs $x_1$ and $x_2$,
but we will specify it using the gates shown.

\begin{figure}[H]
   \begin{center}
      \includegraphics[scale=1]{figures/figure11.png}
   \caption{The light controller circuit.} 
	 \label{fig:11}
	 \end{center}
\end{figure}

The required circuit is described by the Verilog code in Figure~\ref{fig:12}.
Note that the Verilog module is called {\it light} to match the name given in 
Figure~\ref{fig:4}, which was specified when the project was created.
This code can be typed into a file by using any text editor
that stores ASCII files, or by using the Quartus Prime text editing facilities.
While the file can be given any name, it is a common designers' practice to
use the same name as the name of the top-level Verilog module.
The file name must include the extension $v$, which indicates a Verilog
file. So, we will use the name {\it light.v}.

\lstset{language=Verilog}
\begin{figure}[H]
\begin{center}
\begin{lstlisting}
module light (x1, x2, f);
	input x1, x2;
	output f;
	assign f = (x1 & ~x2)|(~x1 & x2);
endmodule 
\end{lstlisting}
\end{center}
\vspace{-0.33in}
	\caption{Verilog code for the circuit in Figure 11.}
	\label{fig:12}
\end{figure}

\subsection{Using the Quartus\textsuperscript{\textregistered}	 Prime Text Editor}

\noindent 
This section shows how to use the Quartus Prime Text Editor.
You can skip this section if you prefer to use some other text editor
to create the Verilog source code file, which we will name {\it light.v}. 

Select {\sf File $>$ New} to get the window in Figure~\ref{fig:13}, 
choose {\sf Verilog HDL File}, and click {\sf OK}. 
This opens the Text Editor window. 
The first step is to specify a name
for the file that will be created. Select {\sf File $>$ Save As}
to open the pop-up box depicted in Figure~\ref{fig:14}. 
In the box labeled {\sf Save as type} choose {\sf Verilog HDL File}.
In the box labeled {\sf File name} type {\it light}.
Put a checkmark in the box {\sf Add file to current project}.
Click {\sf Save}, which puts the file into the directory
{\it introtutorial} and leads to the Text Editor window shown
in Figure~\ref{fig:15}. 
Enter the Verilog code in Figure~\ref{fig:12}
into the Text Editor and save the file by typing {\sf File $>$ Save}, or by typing 
the shortcut {\sf Ctrl-s}.

Most of the commands available in the Text Editor are self-explanatory. 
Text is entered at the {\it insertion point}, which is indicated by a thin
vertical line. The insertion point can be moved either by using the
keyboard arrow keys or by using the mouse. Two features of 
the Text Editor are especially convenient for typing Verilog
code. First, the editor can display different types of Verilog
statements in different colors, which is the default choice. 
Second, the editor can automatically
indent the text on a new line so that it matches the previous line. 
Such options can be controlled by the settings 
in {\sf Tools $>$ Options $>$ Text Editor}.

\begin{figure}[H]
   \begin{center}
      \includegraphics[scale=0.65]{figures/figure13.png}
   \caption{Choose to prepare a Verilog file.} 
	 \label{fig:13}
	 \end{center}
\end{figure}

\begin{figure}[H]
   \begin{center}
      \includegraphics[scale=0.55]{figures/figure14.png}
   \caption{Name the file.} 
	 \label{fig:14}
	 \end{center}
\end{figure}

\begin{figure}[H]
   \begin{center}
      \includegraphics[scale=0.40]{figures/figure15.png}
   \caption{Text Editor window.} 
	 \label{fig:15}
	 \end{center}
\end{figure}

\subsubsection{Using Verilog Templates}

The syntax of Verilog code is sometimes difficult for a
designer to remember. To help with this issue, the Text Editor
provides a collection of {\it Verilog templates}. The templates provide
examples of various types of Verilog statements, such as a {\bf module}
declaration, an {\bf always} block, and assignment statements. 
It is worthwhile to browse through the templates
by selecting {\sf Edit $>$ Insert Template $>$ Verilog HDL} to become 
familiar with this resource.

\subsection{Adding Design Files to a Project}

As we indicated when discussing Figure~\ref{fig:6}, you can tell Quartus Prime software
which design files it should use as part of the current project.
To see the list of files already included in the {\it light} project,
select {\sf Assignments $>$ Settings}, which leads to the window in Figure~\ref{fig:16}.
As indicated on the left side of the figure, click on the item {\sf Files}.
An alternative way of making this selection is to choose
{\sf Project $>$ Add/Remove Files in Project}.

If you used the Quartus Prime Text Editor to create the file and checked
the box labeled {\sf Add file to current project},
as described in Section 5.1, then the {\it light.v}
file is already a part of the project and will be listed in
the window in Figure~\ref{fig:16}.
Otherwise, the file must be added to the project. 
So, if you did not use the Quartus Prime Text Editor, then place a copy of the 
file {\it light.v}, which you created using some other text editor, into 
the directory {\it introtutorial}.
To add this file to the project, click on the {\sf ...} button next to the 
box labeled {\sf File name} in
Figure~\ref{fig:16} to get the pop-up window in Figure~\ref{fig:17}.
Select the {\it light.v} file and click {\sf Open}.
The selected file is now indicated in the {\sf File name} box in Figure~\ref{fig:16}. 
Click {\sf Add} then {\sf OK} to include the {\it light.v} file in the project.
We should mention that in many cases the Quartus Prime software is able to 
automatically find the right files to use for each entity 
referenced in Verilog code, even if the file has not been 
explicitly added to the project. However, for complex projects that
involve many files it is a good design practice to specifically
add the needed files to the project, as described above.

\begin{figure}[H]
   \begin{center}
      \includegraphics[scale=0.55]{figures/figure16.png}
   \caption{Settings window.} 
	 \label{fig:16}
	 \end{center}
\end{figure}

\begin{figure}[H]
   \begin{center}
      \includegraphics[scale=0.65]{figures/figure17.png}
   \caption{Select the file.} 
	 \label{fig:17}
	 \end{center}
\end{figure}
