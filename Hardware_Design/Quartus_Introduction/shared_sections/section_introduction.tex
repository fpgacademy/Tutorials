\section{Introduction}


This tutorial presents an introduction to the Quartus\textsuperscript{\textregistered} Prime 
CAD system.  It gives a general overview of a typical CAD flow for designing circuits that are
implemented by using FPGA devices, and shows how this flow is realized in
the Quartus Prime software. The design process is illustrated by giving step-by-step
instructions for using the Quartus Prime software to implement a very simple circuit
in an Intel\textsuperscript{\textregistered} (Altera) FPGA device.

The Quartus Prime system includes full support for all of the popular methods of
entering a description of the desired circuit into a CAD system. 
Three versions of this tutorial are available; with the Verilog hardware description, with VHDL, 
and with schematic diagram.  This tutorial makes use of the \typeName{} design entry method.  

The last step in the design process involves programming the designed circuit
into an actual FPGA device. To perform this programming step the reader needs to have an FPGA
board connected to their computer, such as the DE-series Development and Education boards that
are described in the \texttt{Teaching and Projects Boards} 
section of the {\small \href{https://www.fpgacademy.org/boards.html} {FPGAcademy.org}} website. 
A reader who does not have access to a DE-series board will still find this part of the
tutorial useful to learn how the FPGA programming and configuration task can be performed.

The screen captures in the tutorial were obtained using the 
Quartus Prime version \versnum \ \edition{} Edition; other versions of the software may be slightly different.
