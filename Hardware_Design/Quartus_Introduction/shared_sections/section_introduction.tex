\section{Introduction}


This tutorial presents an introduction to the Quartus\textsuperscript{\textregistered} Prime CAD system.
It gives a general overview of a typical CAD flow for designing circuits that are
implemented by using FPGA devices, and shows how this flow is realized in
the Quartus Prime software. The design process is illustrated by giving step-by-step
instructions for using the Quartus Prime software to implement a very simple circuit
in an Intel\textsuperscript{\textregistered} FPGA device.

The Quartus Prime system includes full support for all of the popular methods of
entering a description of the desired circuit into a CAD system. This tutorial
makes use of the \typeName{} design entry method, in which the user specifies the desired
circuit in the \typeName{} hardware description language. 
Three versions of this tutorial are available; one
uses the Verilog hardware description language, another uses the VHDL hardware description language, and the third is based on defining
the desired circuit in the form of a schematic diagram.

The last step in the design process involves configuring the designed circuit
in an actual FPGA device. To show how this is done, it is assumed that the user has access
to the Intel DE-series Development and Education board connected to 
a computer that has Quartus Prime software installed. 
A reader who does not have access to the DE-series board will still find the tutorial useful
to learn how the FPGA programming and configuration task is performed.

The screen captures in the tutorial were obtained using the 
Quartus Prime version \versnum \ \edition{} Edition; other versions of the software may be slightly different.
